\documentclass[12pt]{article}

% \usepackage[ngerman]{babel}
\usepackage{multicol}
\usepackage{enumitem}
\usepackage{setspace}
\usepackage{csquotes}
\usepackage{ulem}
\usepackage{amsmath}
% \usepackage{hyperref}  % use if clickable references wished

% pdfLaTeX font configurations
\usepackage[T1]{fontenc}
\usepackage[mono,vvarbb,upint]{notomath}

% % for CJK characters, compile with XeLaTeX
% \usepackage{fontspec}
% \usepackage[vvarbb,upint]{notomath}
% \setmonofont{Noto Sans Mono}
% \usepackage{xeCJK}
% \setCJKmainfont{Noto Serif CJK TC}

\usepackage[height=9in,width=7in,
  top=68pt,headheight=28pt,headsep=20pt,
  heightrounded]{geometry}
\linespread{1.25}
\setlength\parindent{0pt}

\usepackage{fancyhdr}
\pagestyle{fancy}
\fancyhf{}

\fancyhead[L]{Hanxin Xia - 3417418 $\vert$ Vittorio Ciccarelli - 3203477 $\vert$ Anna Stein - 2934420\\Aufgabe 1 im Aufgabenblatt 3}
\fancyhead[R]{QMdCL: Grundlagen des maschinellen Lernens}
\fancyfoot[C]{\thepage}

% customize section title font
\usepackage{titlesec}
\titleformat{\section}{\normalsize\bfseries}{\thesection}{1em}{}

% citation settings
\usepackage{natbib}
\makeatletter
\DeclareRobustCommand\citep
{\begingroup\NAT@swatrue\let\NAT@ctype\z@\NAT@partrue
  \@ifstar{\NAT@fulltrue\NAT@citetp}{\NAT@fullfalse\NAT@citetp}}
\makeatother

\usepackage{qtree}
\usepackage{forest}
\forestset{qtree/.style={baseline, for tree={parent anchor=south,
  child anchor=north,align=center,inner sep=0pt}}}
\useforestlibrary{linguistics}
\forestapplylibrarydefaults{linguistics}

\usepackage{gb4e}
\setlength{\glossglue}{15pt}

%%%%%%%%%%%%%%%%%%%%%%%%%%%%%%%%%%%%%%%%%%%%%%%%%

\begin{document}



\begin{multicols}{2}

\vspace{-3mm}
\begin{math}
  \displaystyle
  x_1 \times x_1: (1, 2) \times (1, 2) = 1 \times 1 + 2 \times 2 = 5\\
  x_1 \times x_2: (1, 2) \times (-2, 1) = 1 \times -2 + 2 \times 1 = 0\\
  x_1 \times x_3: (1, 2) \times (-1, -2) = 1 \times -1 + 2 \times -2 = -5\\
  x_1 \times x_4: (1, 2) \times (2, -1) = 1 \times 2 + 2 \times -1 = 0\\
\end{math}

\begin{math}
  \displaystyle
  x_2 \times x_1: (-2, 1) \times (1, 2) = -2 \times 1 + 1 \times 2 = 0\\
  x_2 \times x_2: (-2, 1) \times (-2, 1) = -2 \times -2 + 1 \times 1 = 5\\
  x_2 \times x_3: (-2, 1) \times (-1, -2) = -2 \times -1 + 1 \times -2 = 0\\
  x_2 \times x_4: (-2, 1) \times (2, -1) = -2 \times 2 + 1 \times -1 = -5\\
\end{math}

\begin{math}
  \displaystyle
  x_3 \times x_1: (-1, -2) \times (1, 2) = -1 \times 1 + -2 \times 2 = -5\\
  x_3 \times x_2: (-1, -2) \times (-2, 1) = -1 \times -2 + -2 \times 1 = 0\\
  x_3 \times x_3: (-1, -2) \times (-1, -2) = -1 \times -1 + -2 \times -2 = 5\\
  x_3 \times x_4: (-1, -2) \times (2, -1) = -1 \times 2 + -2 \times -1 = 0\\
\end{math}

\begin{math}
  \displaystyle
  x_4 \times x_1: (2, -1) \times (1, 2) = 2 \times 1 + -1 \times 2 = 0\\
  x_4 \times x_2: (2, -1) \times (-2, 1) = 2 \times -2 + -1 \times 1 = -5\\
  x_4 \times x_3: (2, -1) \times (-1, -2) = 2 \times -1 + -1 \times -2 = 0\\
  x_4 \times x_4: (2, -1) \times (2, -1) = 2 \times 2 + -1 \times -1 = 5\\
\end{math}

\columnbreak

\null \vfill
\[
\Longrightarrow \qquad
\begin{bmatrix}
  \hphantom{-}5 & \hphantom{-}0 &            -5 & \hphantom{-}0\\
  \hphantom{-}0 & \hphantom{-}5 & \hphantom{-}0 &            -5\\
             -5 & \hphantom{-}0 & \hphantom{-}5 & \hphantom{-}0\\
  \hphantom{-}0 &            -5 & \hphantom{-}0 & \hphantom{-}5
\end{bmatrix}
\]
\vfill \null
\end{multicols}


\textbf{ii}

Lagrange function for a hard margin SVM:

\[
  L(w,b,\alpha) = \frac{1}{2}\Vert w \Vert^2 -\sum_{i=1}^N \alpha_i[y_i (w^Tx_i+b) -1]
\]

Partial derivations for $w, b, \alpha$:

\vspace{-3mm}
\begin{flalign*}
  \frac{\sigma L}{\sigma w} &= w - \sum_{i=1}^N \alpha_i y_i x_i = 0 &&\\
  \frac{\sigma L}{\sigma b} &= - \sum_{i=1}^N \alpha_i y_i = 0 &&\\
  \frac{\sigma L}{\sigma \alpha} &= y_i (w^T x_i + b) - 1 = 0
\end{flalign*}
\newpage

Substitute $w, b$ into the third equation:

\vspace{-3mm}
\begin{flalign*}
  y_i ((\sum_{j=1}^N \alpha_j y_j x_j)^T x_i + \frac{1}{N}\sum_{j=1}^N (\alpha_j y_j x_j)^T x_i) -1 = 0 &&
\end{flalign*}

Calculate  $\alpha$ for $x_1 = [5, 0, -5, 0]$ and $y_1 = 1$ using:

\vspace{-3mm}
\begin{flalign*}
  \alpha_i = \frac{1}{y_i x^T x_i} &&
\end{flalign*}

we get:

\vspace{-3mm}
\begin{flalign*}
  \alpha_1 &= \frac{1}{1 \times (5^2 + 0^2 + (-5)^2 + 0^2)} &&\\
    & ~=~
  \frac{1}{1 \times (25 + 0 + 25 + 0)} &&\\
    & ~=~
  \frac{1}{50}
\end{flalign*}

Similarly, we get $\alpha_2 = \frac{1}{0}$, $\alpha_3 = \frac{1}{50}$, $\alpha_4 = \frac{1}{0}$.

Hence $\alpha* = (0.02, \frac{1}{0}, 0.02, \frac{1}{0})$.


\textbf{iii}

\vspace{-3mm}
\begin{flalign*}
  \alpha_1 &= - 2 \times (\frac{1}{50}  \times 1) = -0.04 &&\\
  \alpha_3 &= - 2 \times (\frac{1}{50}  \times 1) = -0.04
\end{flalign*}

\vspace{-3mm}
\begin{flalign*}
  x_1 &= \sum_{i=1}^2 (\frac{1}{50} \times 1 \times [5, 0, -5, 0] + 0.04) \times 2 &= [0.14,0.04,-0.06,0.04] \times 2 &= [0.28,0.08,-0.12,0.08] &&\\
  x_3 &= \sum_{i=1}^2 (\frac{1}{50} \times 1 \times [-5, 0, 5, 0] + 0.04) \times 2 &= [-1.06,0.04,0.14,0.04] \times 2 &= [-2.12,0.08,0.28,0.08]
\end{flalign*}

















{\setstretch{1.00}
\bibliographystyle{chicago}
\bibliography{references}}

\end{document}
