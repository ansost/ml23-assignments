\documentclass[12pt]{article}

% \usepackage[ngerman]{babel}
\usepackage{multicol}
\usepackage{enumitem}
\usepackage{setspace}
\usepackage{csquotes}
\usepackage{ulem}
\usepackage{amsmath}
% \usepackage{hyperref}  % use if clickable references wished

% pdfLaTeX font configurations
\usepackage[T1]{fontenc}
\usepackage[mono,vvarbb,upint]{notomath}

% % for CJK characters, compile with XeLaTeX
% \usepackage{fontspec}
% \usepackage[vvarbb,upint]{notomath}
% \setmonofont{Noto Sans Mono}
% \usepackage{xeCJK}
% \setCJKmainfont{Noto Serif CJK TC}

\usepackage[height=9in,width=7in,
  top=68pt,headheight=28pt,headsep=20pt,
  heightrounded]{geometry}
\linespread{1.25}

\usepackage{fancyhdr}
\pagestyle{fancy}
\fancyhf{}

\fancyhead[L]{Hanxin Xia - 3417418 $\vert$ Vittorio Ciccarelli - 3203477 $\vert$ Anna Stein - 2934420\\Aufgabe 1 im Aufgabenblatt 3}
\fancyhead[R]{QMdCL: Grundlagen des maschinellen Lernens}
\fancyfoot[C]{\thepage}

% customize section title font
\usepackage{titlesec}
\titleformat{\section}{\normalsize\bfseries}{\thesection}{1em}{}

% citation settings
\usepackage{natbib}
\makeatletter
\DeclareRobustCommand\citep
{\begingroup\NAT@swatrue\let\NAT@ctype\z@\NAT@partrue
  \@ifstar{\NAT@fulltrue\NAT@citetp}{\NAT@fullfalse\NAT@citetp}}
\makeatother

\usepackage{qtree}
\usepackage{forest}
\forestset{qtree/.style={baseline, for tree={parent anchor=south,
  child anchor=north,align=center,inner sep=0pt}}}
\useforestlibrary{linguistics}
\forestapplylibrarydefaults{linguistics}

\usepackage{gb4e}
\setlength{\glossglue}{15pt}

%%%%%%%%%%%%%%%%%%%%%%%%%%%%%%%%%%%%%%%%%%%%%%%%%

\begin{document}

\noindent \textbf{i}

\begin{multicols}{2}
\noindent
\begin{math}
  \displaystyle
  x_1 \times x_1: (1, 2) \times (1, 2) = 1 \times 1 + 2 \times 2 = 5\\
  x_1 \times x_2: (1, 2) \times (-2, 1) = 1 \times -2 + 2 \times 1 = 0\\
  x_1 \times x_3: (1, 2) \times (-1, -2) = 1 \times -1 + 2 \times -2 = -5\\
  x_1 \times x_4: (1, 2) \times (2, -1) = 1 \times 2 + 2 \times -1 = 0\\
\end{math}

\noindent
\begin{math}
  \displaystyle
  x_2 \times x_1: (-2, 1) \times (1, 2) = -2 \times 1 + 1 \times 2 = 0\\
  x_2 \times x_2: (-2, 1) \times (-2, 1) = -2 \times -2 + 1 \times 1 = 5\\
  x_2 \times x_3: (-2, 1) \times (-1, -2) = -2 \times -1 + 1 \times -2 = 0\\
  x_2 \times x_4: (-2, 1) \times (2, -1) = -2 \times 2 + 1 \times -1 = -5\\
\end{math}

\noindent
\begin{math}
  \displaystyle
  x_3 \times x_1: (-1, -2) \times (1, 2) = -1 \times 1 + -2 \times 2 = -5\\
  x_3 \times x_2: (-1, -2) \times (-2, 1) = -1 \times -2 + -2 \times 1 = 0\\
  x_3 \times x_3: (-1, -2) \times (-1, -2) = -1 \times -1 + -2 \times -2 = 5\\
  x_3 \times x_4: (-1, -2) \times (2, -1) = -1 \times 2 + -2 \times -1 = 0\\
\end{math}

\noindent
\begin{math}
  \displaystyle
  x_4 \times x_1: (2, -1) \times (1, 2) = 2 \times 1 + -1 \times 2 = 0\\
  x_4 \times x_2: (2, -1) \times (-2, 1) = 2 \times -2 + -1 \times 1 = -5\\
  x_4 \times x_3: (2, -1) \times (-1, -2) = 2 \times -1 + -1 \times -2 = 0\\
  x_4 \times x_4: (2, -1) \times (2, -1) = 2 \times 2 + -1 \times -1 = 5\\
\end{math}

\columnbreak

\null \vfill
\[
\Longrightarrow \qquad
\begin{bmatrix}
  \hphantom{-}5 & \hphantom{-}0 &            -5 & \hphantom{-}0\\
  \hphantom{-}0 & \hphantom{-}5 & \hphantom{-}0 &            -5\\
             -5 & \hphantom{-}0 & \hphantom{-}5 & \hphantom{-}0\\
  \hphantom{-}0 &            -5 & \hphantom{-}0 & \hphantom{-}5
\end{bmatrix}
\]
\vfill \null
\end{multicols}


\vspace{20pt}

\noindent \textbf{ii} --- anotherthing anotherthing \bigbreak




{\setstretch{1.00}
\bibliographystyle{chicago}
\bibliography{references}}

\end{document}
